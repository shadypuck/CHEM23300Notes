\documentclass[../notes.tex]{subfiles}

\pagestyle{main}
\renewcommand{\chaptermark}[1]{\markboth{\chaptername\ \thechapter\ (#1)}{}}
\setcounter{chapter}{6}

\begin{document}




\chapter{Bulk Transport}
\section{Exocytosis and Endocytosis}
\begin{itemize}
    \item \marginnote{11/8:}Two parts of lecture today:
    \begin{itemize}
        \item How a protein that is formed in the ER reaches the plasma membrane.
        \item How proteins in the plasma membrane get to other parts of the cell (endocytotic pathway/other plasma membrane locations).
    \end{itemize}
    \item Krishnan has really enjoyed teaching this class :)
    \item ER to Golgi transport.
    \begin{itemize}
        \item The golgi is the pathway to the plasma membrane.
        \item We don't know too much about the Golgi.
        \item Glycobiology.
        \item Will become one of the most important organelles in the future because it's very important and we don't know that much about it.
        \item The Golgi is very hard to model (it's a stack of pancakes, and these are hard to distinguish).
        \item Proteins that go to the plasma membrane and lysosomes and get secreted all have to pass through the golgi.
    \end{itemize}
    \item COPI and COPII coated vesicles.
    \begin{itemize}
        \item How a protein starts its journey.
        \item If there's a protein in the ER lumen, there's a massive sugar transferred from dolichol to a particular asparagine.
        \item Every protein that get's secreted gets so labeled.
        \item Don't worry about what happens to that sugar rn, but it's like an assembly line (different steps of the packing process, all delocalized).
        \item If you have a huge amount of transport proteins, you're going to have errors (things getting sent out that shouldn't), so you need a way to bring them back.
        \item When the ER buds, we have to concentrate the proteins inside it.
        \item The vesicle is covered in a protein called COPII (the COPII complex) that signals it as outbound from the ER.
        \item ER-Golgi intermediate compartment does exist.
        \item Proteins that shouldn't have left get sent back (in vesicles coated with COPI).
        \item Anything in the ER (most proteins that need to be secreted) need to fold.
        \begin{itemize}
            \item There are chaperones in the ER (such as BIP, calnexin) that help other proteins fold.
            \item Many big proteins (90\% of them) don't fold properly, so proteins need ways to make sure that they're only excreting the right proteins.
            \item The ER is host to many unfolded proteins. Chaperones recognize anything that is misfolded and mark them for degradation.
        \end{itemize}
        \item How do proteins get into the vesicle?
        \begin{itemize}
            \item Proteins have an exit signal that interacts with a cargo receptor.
            \item Cargo receptors cluster on the surface.
        \end{itemize}
        \item Now the patch buds out.
        \item These receptors usually have a lumenal domain and a cytosolic domain.
        \item Adapter proteins recognize a bound form and cause proteins to cluster. COPII then assembles on the outside of the vesicle.
    \end{itemize}
    \item The basics of vesicular fusion.
    \begin{itemize}
        \item A lot of fission and fusion today.
        \begin{itemize}
            \item Big in neuroscience and neurobiology, but also occur in cell biology.
            \item These occur because of t-SNAREs and v-SNAREs (allow vesicles to fuse).
        \end{itemize}
        \item Vesicle fusion: \textbf{Homotypic fusion} and \textbf{heterotypic fusion}.
        \begin{itemize}
            \item You also have \textbf{N-ethylmaleimide sensitive factor}.
        \end{itemize}
    \end{itemize}
    \item \textbf{Homotypic fusion}: The fusion of two like membranes.
    \begin{itemize}
        \item E.g., two lysosomes fusing to form a bigger lysosome.
    \end{itemize}
    \item \textbf{Heterotypic fusion}: The fusion of two different kinds of membranes.
    \begin{itemize}
        \item E.g., a COPI-coated vesicle and the ER.
    \end{itemize}
    \item \textbf{Vesicular SNARE}: Occurs on every organelle. \emph{Also known as} \textbf{v-SNARE}.
    \item \textbf{Target SNARE}: The specifying SNARE. \emph{Also known as} \textbf{t-SNARE}.
    \item \textbf{N-ethylmaleimide sensitive factor}: \emph{Also known as} \textbf{NSF}.
    \begin{itemize}
        \item Pries apart t-SNAREs and v-SNARES, allowing fusion.
        \item The two helical domains will then move apart.
    \end{itemize}
    \item KDEL is an ER-retrieval sequence.
    \begin{itemize}
        \item How do we send proteins that accidentally localized to vesicles back?
        \item Original way out:
        \item Bulk-phase endocytosis: Letting the vesicle fill up naturally and the sending it back; not specific.
        \item Alternatively, you can attract your proteins to the future vesicle (receptor-mediated endocytosis).
        \item The purpose of the receptor is to concentrate your substance.
        \item How does cargo that's exited this way get sent back?
        \item KDEL receptors bind the sequence KDEL in a protein. A bit in the golgi, but most of it is in the ER. But its function is to grab ER retrieval sequences on the N- or C-terminus. pH dependence.
        \item How will an ER retrieval sequence differ from a localization sequence?
        \begin{itemize}
            \item The fundamental difference is one is regulated, and the other is spontaneous.
        \end{itemize}
        \item What we still don't know: How does a vesicle get switched from a COPII coat to a COPI coat?
    \end{itemize}
    \item Spatial position of the Golgi in the animal and plant cell.
    \begin{itemize}
        \item 5-6 stacks in human cells.
        \item 100-200 stacks in plant cells (plant cells have to secrete a huge amount of material to maintain a cell wall).
        \item More stacks (cisterni) lead to a more advanced golgi.
        \item Cisterni are well below the wavelength of light, so we need electron microscopy.
    \end{itemize}
    \item Molecular compartmentalization of the Golgi apparatus.
    \begin{itemize}
        \item Our protein gets dolichol-marked.
        \item Different reactions in different cisterna.
        \item Different enzymes in each cisterna.
        \item Each location in the Golgi acts on the protein differently before its eventually secreted due to the difference in enzymes stored in each region.
        \item Cis-Golgi: beginning of the Golgi (closer to ER).
        \item Stack.
        \item Then trans-Golgi (end; opposite side from the ER).
        \item Krishnan goes over an experiment showing what's localized in what compartment (different tagging proteins tag different enzymes, revealing localization in an electron micrograph).
    \end{itemize}
    \item Oligosaccharides processing in the Golgi.
    \begin{itemize}
        \item N-acetylglucosamine (GLcNAc).
        \item Mannose (Man).
        \item Galactose (Gal).
        \item Sialic acid (NANA).
        \item These all get attached. How far along a cargo has gone is decided by the sugar ordering.
        \item Initial trimming of mannose and glucose. Trimming takes off a lot of these sugars, and then we replace with specific sugars (specifically those 4 above).
        \item When a protein/lipid that's gone through the entire process reaches the plasma membrane, it will be able to show 2-types of sugar: Complex oligosaccharides (with a high concentration of negatively charged sialic acid at the end) and high-mannose oligosaccharides that do not get sialic acid added because the sugars arranged on the protein are inaccessible.
    \end{itemize}
    \item What is the purpose of glycosylation?
    \begin{itemize}
        \item Most important slide of the first part of the lecture!
        \item In the ER lumen, we have two initial enzymes (glucosidase I and II) that chop off glucoses.
        \item Manosidase in the ER takes away mannoses.
        \item Leaves behind a sugar that's good to go to the Golgi.
        \item Golgi mannosidase takes off 3 mannose residues at a time (the accessible ones).
        \item In the medium golgi: We start adding GlcNAc, then add galactose, then silylation.
        \item GlcNAc is added by N-acetylglucosamine transferase I.
        \item Something attached to gludine, which is a good leaving group (highly anionic).
        \item All these glycosylated molecules are present inside the comparment.
        \item How does localization happen?
        \begin{itemize}
            \item We have membrane proteins that will take in particular molecules and will work with them in one particular compartment.
        \end{itemize}
        \item Another mannosidase event (Golgi mannosidase II).
        \item Now proteins are Endo-H resistant.
        \item Addition of 2 more GlcNAc molecules, then galactose, then silylation. These give us our complex oligosaccharide.
        \item The rules of the molecule Endo-H (of bacterial origin) tells you how far a sugar has gone in its journey. Helped us figure out many enzymes and transporters involved in the pathway.
        \item You can work out the molecular weight of a protein.
        \item High gets sent one place, low gets sent another place. This helped us investigate stuff.
    \end{itemize}
    \item Two models of Golgi-protein transport.
    \begin{itemize}
        \item We still don't know this (it's being researched, largely at UChicago).
        \item What is the model of protein transport in the Golgi?
        \item Does a \emph{cis}-Golgi gradually mature (Golgi cisterni "grow up", gaining/losing proteins along the way), or do we have proteins transferred between fixed cisterni. The other one has vesicles budding out to either go forward to the next cisterni or back to the previous cisterna.
        \item \textbf{Cisternal maturation model} vs. \textbf{vesicle transport model}.
        \item If you want to know which is currently winning, write to UChicago's Ben Glick :)
    \end{itemize}
    \item What keeps the golgi together (why are all of our pancakes stuck together)?
    \begin{itemize}
        \item There are hydrophobic tentacles called golgins that wind together and prevent cytosolic fluid from getting between the pancakes.
        \item At the time of cell-division or during apoptosis, we need to disentegrate the golgi because we can't have ?? hanging around.
        \item This is induced by a kinase which phosphorylates the golgins, causing fragmentation.
        \item Once the cell membrane comes back, the golgi reform and you get new ones.
        \item When a cell divides, its organelles must divide, too, and we only have one golgi.
        \item We build a new cisterni atop the old one.
    \end{itemize}
    \item Onwards and outwards: Exocytosis and secretion.
    \begin{itemize}
        \item There is a basal level of secretion that happens all the time, and there is regulated secretion where you have to release a massive amount of something all at the same time.
        \begin{itemize}
            \item Regulated example: Insulin.
        \end{itemize}
        \item What needs to be secreted \textbf{basally} (all the time)? Mucus!
    \end{itemize}
    \item Protein sorting at the TGN.
    \begin{itemize}
        \item Once something comes to the trans-golgi network, where can it go? To the lysosome, outside the cell, and constitutive (e.g., placing things in the cellular membrane).
    \end{itemize}
    \item Secretory vesicle maturation.
    \begin{itemize}
        \item Very important!
        \item How do we send out a huge amount of glucose, or melanin?
        \item We concentrate molecules/proteins into dense core secreted granules (100-200 nm), extremely high concentration.
        \item Longer peptide has a secretion clock. At each point, you have a pausing condition.
        \item You take advantage of processing to pack tight.
        \item Recall the tight junction from the first lecture (intestinal cell). Different parts of a membrane have different properties. Non-leaking proteins. You need a way to get proteins to exactly one part of the plasma membrane (how this works still isn't understood very well).
    \end{itemize}
    \item Sorting plasma membrane proteins in polarized cells.
    \begin{itemize}
        \item Two models: Direct and indirect sorting.
        \item One model is different vesicles go to different locations. The second is you get random input into the plasma membrane, and you then have sorting at the endocytotic level with the help of an endosome that takes ones in the wrong place to the right place.
        \item One recycling endosome goes to the cell surface, the other one elsewhere.
    \end{itemize}
    \item Golgi to lysosome transport.
    \begin{itemize}
        \item Lysosome proteins are all set up for proteolysis (chopping things up) and glycosidases. Take old cellular machinery, chop it up into its component parts, and let it get recycled.
        \item A cell \emph{must} recycle stuff, or it will need to make so much more amino acids.
        \item Lysosomes are \emph{highly} acidic and contain degratory proteins.
        \item Why don't lysosomes digest themselves?
        \item pH 5 and the proteins still work. Usually this pH would denature the protein, but instead lysosome proteins are built for this.
        \item Hydrolases chop stuff up; they've evolved to withstand the highly acidic environment.
        \item Mannose phosphate receptor in the Golgi: A bus to the lysosome and back (takes hydrolases from the trans-golgi to the pre-lysosomal compartments). Not a lysosome resident protein. Why doesn't it get denatured? Glycosylation of the receptor protects it from the hydrolase and the lysosome; can't purely operate at pH 5 but has to be stable at multiple.
        \item Glycosylation tree branches around a specific protein protect it from the actions of the lysosome.
    \end{itemize}
    \item Transport of newly synthesized lysosomal hydrolases to endosomes.
    \begin{itemize}
        \item The mannose phosphate receptor is our bus.
        \item We have a standard lysosomal hydrolase.
        \item Carries a mannose as it enters the Golgi.
        \item Addition of phospho-GlcNAc allows us to put a phosphate onto manose, and then it goes away. It's like a cofactor. Then it gets cut off.
        \item Protonation allows the protein to fall off and enter the lysosome.
        \item Then a retromer coat allows our receptor to be retrieved.
    \end{itemize}
    \item Recognition of a lysosomal hydrolase.
    \begin{itemize}
        \item Start with a lysosomal hydrolase carrying a glycosylation on the N-terminal .
        \item GlcNAc phosphotransferase transfers a phosphate onto the lysosomal
        \item UDP-GLcNAc transfers a phosphate and a GlcNAc to the lysosomal hydrolase, kicking out UMP.
        \item Then the enzyme releases the lysosomal hydrolase. We then remove the GlcNAc, leaving mannose 6-phosphate behind (M6P).
    \end{itemize}
    \item Ways to enter the lysosome.
    \begin{itemize}
        \item Phagocytosis: Take in a bacteria to test it; use the lysosome to break it down into pieces that can be used by the rest of the cell (e.g., for defense).
        \item Endocytosis: Take things in from the outside and digest them.
        \item Autophagy: We automatically create a vesicle around something inside and move it to the lysosome.
    \end{itemize}
    \item Endocytosis.
    \begin{itemize}
        \item We take stuff in from the outside to an early endosome (vesicle within a cell; an intracellular sorting organelle).
        \item Microtubule mediated transport to wherever we need, e.g., the lysosome, the trans-Golgi network, etc.
    \end{itemize}
    \item Different mechanisms of endocytosis.
    \begin{itemize}
        \item Macropinocytosis: An appendage sticks out and closes in.
        \item Clathrin-coated vesicle: ...
        \item Noncoated vesicle: ...
        \item Caveolae: ...
        \item Phagocytosis: ...
    \end{itemize}
    \item Clathrin-mediated endocytosis.
    \begin{itemize}
        \item Clathrin shapes rounding.
        \item Makes things go to specific plases.
        \item Receptor-mediated endocytosis.
    \end{itemize}
    \item Receptor-mediated endocytosis.
    \begin{itemize}
        \item Receptors on the plasma membrane draw external stuff into the cell.
    \end{itemize}
    \item Recycling endosomes.
    \begin{itemize}
        \item We don't want to destroy things we'll use again.
        \item Membrane proteins that aren't needed go to an endosome, and when they're needed, they bud off and go back to the membrane.
    \end{itemize}
    \item Degrading proteins: Autophagy.
    \begin{itemize}
        \item Nucleation and extension: Bits of phospholipid engulf cytosol and organelles.
        \item Transport to the lysosome.
        \item Digestion: There, acid hydrolases break down the material.
    \end{itemize}
    \item Transcytosis.
    \begin{itemize}
        \item Moving something through a cell and to the other side. Think intestinal cells.
        \item Endocytosis on the one side. Transport to the early entosome. Multiple pathways from there.
        \item We can have an empty transport vesicle go back to the plasma membrane to replenish the phospholipidds there.
        \item A full one can bud off to go for degradation in the endolysosome.
        \item A full one can bud off and go to a recycling endosome for transcytosis to the far cell wall.
    \end{itemize}
    \item Exocytosis.
    \begin{itemize}
        \item Needed for cytokinesis, phagocytosis, plasma membrane repair, and cellularization.
    \end{itemize}
\end{itemize}



\section{Office Hours (Krishnan)}
\begin{itemize}
    \item Exam format:
    \begin{itemize}
        \item There will be about 10 questions on the paper, so about 12 mins/question.
        \item 15/65 points on protein transport and localization.
        \item Some questions on sequencing.
        \item There will be one hydropathy index plot question.
        \item Review strategies: Go through the four lectures that have been taught.
        \item The guest lecture will not be covered at all on the midterm.
    \end{itemize}
    \item What is oxidative stress?
    \begin{itemize}
        \item Not Krishnan's question; won't be testable.
    \end{itemize}
    \item What primer do you create if you haven't sequenced yet?
    \begin{itemize}
        \item If you want to sequence my CFPR gene, for example, we have the whole human genome at this point in time, so you can assume that mine will be similar. Thus, we design based on past precedent.
        \item If you want to do whole-genome sequencing, you sonicate your DNA and don't know what starts and ends where. In this case, use an adapter to specifically target certain parts.
    \end{itemize}
    \item What are these adapters?
    \begin{itemize}
        \item There are many ways to attach an adapter. You can use a poly A polymerase which adds tons of adenines to the end and then adapt to that.
        \item You can also use ligase to attach the $5'$ end of your adapter to the $3'$ end of your DNA duplex. Similar if you have a $3'$ adapter.
    \end{itemize}
    \item What does it mean that apyrase is an eraser?
    \begin{itemize}
        \item An apyrase converts ADP to AMP. It is basically a reset mechanism that allows you to add the next nucleotide.
        \item Recall that in pyrosequencing, you have a bead with clonal DNA copies all over it. Dumping in a certain dNTP...
    \end{itemize}
    \item How do the variable lengths in PCR come in? How do they eventually become the same length?
    \begin{itemize}
        \item See the videos on Canvas.
    \end{itemize}
    \item In Maxam-Gilbert sequencing, is the strand cleaved before or after the modification.
    \begin{itemize}
        \item Chemically, each modified nucleotide is significantly altered during the reaction, and the phosphate backbone breaks on both sides of it. You will see this in PSet 3, Q2a.
        \item Thus, it is best if we describe cleavage as happening "before" the modification, since everything after the last base  before the modification will either be cleaved away or entirely destroyed.
    \end{itemize}
\end{itemize}




\end{document}