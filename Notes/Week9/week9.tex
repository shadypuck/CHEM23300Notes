\documentclass[../notes.tex]{subfiles}

\pagestyle{main}
\renewcommand{\chaptermark}[1]{\markboth{\chaptername\ \thechapter\ (#1)}{}}
\setcounter{chapter}{8}

\begin{document}




\chapter{Protein Engineering}
\section{Protein Post-Translational Modification}
\begin{itemize}
    \item \marginnote{11/29:}Review from last time.
    \begin{itemize}
        \item What you need for unnatural amino acid incorporation: TGA (amber stop codon), special tRNA to both selectively bind the unnatural amino acid and TGA.
    \end{itemize}
    \item We now study work by Wang and Schultz (2001).
    \item tRNA selection.
    \begin{itemize}
        \item If we do negative selection (correct read through generates toxic barnase), nonfunctional tRNA will cause the bacteria to survive.
        \item If the tRNA is only charged by E. coli synthesis, the bacteria will die.
        \item If tRNA is only charged by mj syn, your bacteria will die.
        \item If the tRNA can be charged by both, the bacteria will die.
        \item Now for positive selection: Nonfunctional tRNA? No survival gene, so bacteria dies.
        \item But this time, tRNA only charged by mj syn survives! Thus, if you do negative selection and then positive selection, you will select specifically for what you want.
        \item We can also do positive selection first and then negative selection. In this case, three bacteria survive first, and then two get eliminated.
        \item Positive selection seldom is the opposite of negative selection; negative does specificity, positive does ??
        \item This is a very testable topic.
    \end{itemize}
    \item An orthogonal tRNA.
    \begin{itemize}
        \item You now have a tRNA that can only be charged by exogenous amino acid residues.
        \item Orthogonal tRNA and the right gene gives you great ampicillin resistance
    \end{itemize}
    \item An orthogonal aminoacyl-tRNA synthetase.
    \begin{itemize}
        \item Change the specificity of \emph{Mj} TyrRS so that it charges the selected tRNA with O-methyl-l-tyrosine.
        \item The crystal structure had been determined for the homologous TyrRS from \emph{Bacillus stearothermophilus} bound to a free tyrosine residue.
        \item Five active site residues were found to be within \SI{6.5}{\angstrom} of the \emph{para}-hydroxyl (yellow).
    \end{itemize}
    \item Aminoacyl tRNA synthetase selection.
    \begin{itemize}
        \item Similarly to before, it doesn't matter here if you do positive or negative selection first.
        \item If you get read through for the Cm gene (chloramphenicol resistance), the bacteria survives.
        \item Survivors contain aaRSs capable of charging any natural or unnatural aa onto the orthogonal tRNA.
        \item We don't include the unnatural amino acid in solution for negative selection. Thus, when we have an orthogonal tRNA that only accepts O-MeTyr, these bacteria will survive because they cannot synthesize barnase. Any tRNA that incorporates an\emph{other} amino acid will synthesize barnase and die.
    \end{itemize}
    \item A mutant tRNA synthetase.
    \begin{itemize}
        \item A mutant synthetase was selected byat selectively charged the \emph{Mj} suppressor tRNA with O-methyl-L-tyrosine.
        \item The substitution removes a hydrogen bond, creating a hydrophobic pocket.
    \end{itemize}
    \item O-MeTyr as UAA incorporation into DHFR.
    \begin{itemize}
        \item Dihydrofolate reductase (DHFR) was generated with TAG in place of the third codon and purified by metal-affinity chromatography.
        \item Tandem MS of DHFR tryptic digest unambiguously shows complete incorporation of O-methyl-L-tyrosine in the third amino acid position.
    \end{itemize}
    \item UAAs incorporated \emph{in vivo}.
    \begin{itemize}
        \item Reactive amino acids for further post-translational modification.
        \item Caged (photo-activable) amino acids to study kinetics and mechanisms.
        \item Heavy atom-labeled amino acids for structural elucidaiton.
        \item Incorporated in \emph{E. coli}, yeast, ...
        \item Incorporating a bioorthogonal handle (alkyne) can be done.
        \item Heavy-carbon side chain for cross-linking is possible.
    \end{itemize}
    \item Recoding \emph{E. coli} genome.
    \begin{itemize}
        \item ...
    \end{itemize}
    \item Summary.
    \begin{itemize}
        \item Strategies to increase protein functional diversity.
        \begin{itemize}
            \item Chemical approaches.
            \item Site-drirected mutagenesis: Changing one amino acid to another.
            \item Unnatural amino acid incorporation. 
        \end{itemize}
        \item Non-sense suppression \emph{in vitro} and \emph{in vivo}.
        \item Orthogonal tRNA and aminoacyl tRNA synthetase.
        \begin{itemize}
            \item Directed evolution (negative and positive).
        \end{itemize}
        \item Applications of unnatural amino acid incorporation.
    \end{itemize}
    \item We now start on protein post-translational modification.
    \item Humans are the most complex organisms on earth. But where does that complexity come from?
    \item Gene number and genome size.
    \begin{itemize}
        \item We have 3 billion base pairs in our genome, containing 20k-25k protein coding genes. These constitute about 1\% of our genome.
        \item Do we have the highest number of genes? No! The most common species of water flea (tiny and translucent) has more than 31k genes (8000 more than us). It has so many genes due to extensive gene duplication, even though we're more complex. We also have this, but it has it more.
        \item \emph{C. elegans} is a model organism that is often used, especially in aging studies since it contains all of the human aging genes. It's life cycle is just much shorter and therefore easier to study. It has a similar amount of protein-coding genes to humans!
        \item Do we have the largest genome by base pairs? The Japanese flower \emph{Paris japonica} has a 149 billion base pair genome, 50 times the size of our haploid genome.
        \item So why are we so complex?
        \item We are complex due to meticulous regulation of transcription and translation (the other 99\% that's not non-coding is largely regulatory; we don't fully understand it, but we know it's not just junk).
        \item Proteome complexity also plays a role.
    \end{itemize}
    \item \textbf{Proteome}: The entire set of proteins that is, or can be, expressed by a genome, cell, tissue, or organism at a certain time.
    \item \textbf{Proteomics}: The study of the proteome, such as proteome profiling.
    \item There's genome, proteome, \textbf{transcriptome}, \textbf{kinome} (all kinases), \textbf{methylome} (all methylated DNA), etc.
    \item \textbf{Kinome}: The complete set of protein kinases encoded in its genome.
    \item Post-translational modifications augment proteome complexity.
    \begin{itemize}
        \item Chemical changes that happen after translation.
        \item Lactylation, other example??
        \item Common types: Phosphorylation, hydroxylation, glycosylation (N-glycosylation from Dr. Krishnan; super important for cell recognition; Dr. Tang use to have a lecture on it), Lysine acetylation, lysine ubiquitinylation (adding a 76-77 aa peptide instead of a small functional group; usually precedes degradation).
        \item You don't have to memorize these; if tested, we will be given a chemical structure.
    \end{itemize}
    \item Protein phosphorylation.
    \begin{itemize}
        \item Catalyzed by kinases.
        \item Within the kinase active site, you have a base that deprotonates the hydroxyl group, leading the \ce{O-} nucleophile to attack the $\gamma$ phosphate in an ATP via nucleophilic acyl substitution.
        \item Protein phosphatase removes phosphates from phosphorylated protein residues, resulting in...
    \end{itemize}
    \item Functions of protein phosphorylation.
    \begin{itemize}
        \item 1/3-2/3 of the proteome in eukaryotes can be phosphorylated.
        \item Not all of this leads to activity (as far as we know at this point).
        \item Phosphorylation alters...
    \end{itemize}
    \item The human kinome comprises 518 kinases.
    \begin{itemize}
        \item Tyrosine kinases have their own corner of the evolutionary phylogenetic tree.
        \item There are 48 FDA approved kinase inhibitors; more than half are for tyrosine kinases.
        \item What's the difficulty? Do we want to inhibit one tyrosine kinase or many? Probably just one, belonging to one protien. Thus, \emph{specificity} is the largest challenge of developing novel kinase inhibitors.
        \item Gleevic (or Imatinib) is a drug for Leukemia. Once the...
    \end{itemize}
    \item MAP kinase pathways: An example of kinase cascade complexity.
    \begin{itemize}
        \item MAPKKKK lol.
        \item One protein can be phosphorylated by many kinases.
    \end{itemize}
    \item Mass spectrometry-based \textbf{phosphoproteomics}.
    \begin{itemize}
        \item ...
        \item There are about 200 phosphorylation sites in the human genome.
        \item We don't know the functions of most phosphorylation sites.
    \end{itemize}
    \item \textbf{Phosphoproteomics}: A branch of proteomics that characterizes phosphorylated proteins.
    \item How can cellular substrates of specific kinases be identified?
    \begin{itemize}
        \item \emph{In vitro} peptide microarray screens.
        \item You take a chip, divide it into 1000-2000 wells.
        \item Each well encodes a specific peptide sequence.
        \item Then you flow on the kinase you want to study and supply radioactive \ce{{}^32P}. Then you can take a radiograph of the chip.
        \item No longer widely used because it's not super biologically relevant.
    \end{itemize}
    \item Tyrosine kinases.
    \begin{itemize}
        \item Shokat's bump and hole strategy for kinase substrate identification.
        \item His lab developed one of the first covalent drugs for ?? inhibition (cancer related).
        \item Make ATP larger so that...
    \end{itemize}
    \item Interaction between mutated kinase and modified ATP.
    \begin{itemize}
        \item ...
    \end{itemize}
    \item Kinase/ATP analog engineering allows substrate tagging in cell lysates but not in live cells.
    \begin{itemize}
        \item ...
    \end{itemize}
    \item Use chemical tagging...
\end{itemize}



\section{Bioorthogonal Chemistry}
\begin{itemize}
    \item \marginnote{12/1:}Nobel prize in Chemistry (2022).
    \item Outline.
    \begin{itemize}
        \item Bioorthogonal chemistry: What and why?
        \item Bioorthogonal reactions.
        \item Applications of bioorthogonal chemistry.
    \end{itemize}
    \item \textbf{Bioorthogonality}: Chemistry inert to the conditions in the particular physiological condition.
    \begin{itemize}
        \item Non-native reactants and selective reactions under physiological conditions.
    \end{itemize}
    \item Understanding biology through chemistry.
    \begin{itemize}
        \item \textbf{Chemical biology} defined.
        \item Key to the field is selectivity.
        \begin{itemize}
            \item Chemical biologists hope to develop molecules with perfectly selective biological function.
            \item In practice, though, we start with as much as we can get on a first attempt and refine from there.
        \end{itemize}
        \item The ability to make chemical modifications that enable direct detection of, or interaction with, biomolecules in their native cellular environments is at the heart of chemical biology.
    \end{itemize}
    \item \textbf{Chemical biology}: The creation of nonbiological molecules that exert an effect on, or reveal new information about, biological systems.
    \item Advantages of bioorthogonal chemistry.
    \begin{itemize}
        \item Applicable to all biomolecules (in theory).
        \item Small size (non-perturbing; better access to intracellular and extravascular compartments).
        \item Bioorthogonality = selectivity (highly selective, low background labeling \emph{in vivo} --- in whole organisms!).
        \item Versatile and divergent (two-step labeling enables various functionalizations of the same reporter group).
    \end{itemize}
    \item Bioorthogonal reactions.
    \item Requirements of bioorthogonal chemistry.
    \begin{itemize}
        \item Definition of a \textbf{bioorthogonal reaction}.
    \end{itemize}
    \item \textbf{Bioorthogonal reaction}: A chemical reaction that satisfies the following conditions.
    \begin{enumerate}
        \item The reaction must be chemically selective and compatible with an aqueous environment.
        \item Reaction yields a stable covalent linkage without toxic byproducts.
        \item Reactnats must be kinetically, thermodynamically, and metabolically stable, and nontoxic prior to reaction.
    \end{enumerate}
    \item Important reactions:
    \begin{itemize}
        \item Ketone condensation, e.g., with a labeled hydroxime.
        \item Staudinger ligation, with an aza-ylide (nitrogen and phosphorous interacting).
        \item CuAAC (see Figure \ref{fig:clickRxn}).
        \item SPAAC (same idea as the above, except the alkyne is in a ring).
        \item Tetrazine-BCN, with a 4x N ring that loses \ce{N2} and bonds with an cycloalkene instead.
        \item Different rate constants for various reactions shown.
    \end{itemize}
    \item Ketone condensation.
    \begin{itemize}
        \item Reactions of aldehydes/ketones with amine nucleophiles (usually hydrazine/alkoxyamine).
        \item Usually requires acidic pH, slow kinetics, mM concentration of reagent, competition from endogenous or naturally occuring aldehydes and ketones.
        \item Aniline accelerates the reaction > 40-fold (>400-fold at $\pH=4.5$).
        \item Pictet-Spengler ligation of the aldehyde with an alkoxyamine derivative is best.
    \end{itemize}
    \item Staudinger ligation.
    \begin{itemize}
        \item Azide reacts with \ce{RPPh2} under mild conditions.
        \item Internal electrophilic trap forms amide linkage.
        \item Phosphines are relatively unreactive toward biological functional groups.
        \item Reaction is relatively slow.
    \end{itemize}
    \item Cu(I)-catalyzed azid-alkyne cycloaddition (CuAAC).
    \begin{itemize}
        \item Azide (1,3-dipole) can undergo reactions iwth activated alkynes.
        \item Forms triazole products but at physiological conditions.
        \item Fast, but high cellular toxicity.
    \end{itemize}
    \item Strain-promoted $[3+2]$ cycloaddition.
    \begin{itemize}
        \item Strain-promoted azide-alkyne cycloaddition (SPAAC).
        \item Catalyst free $[3+2]$ (toxic Cu(I) not needed).
        \item Can be performed on the surface of living cells.
        \item Increase reaction rate with the addition of an EWG on cyclooctyne.
    \end{itemize}
    \item Nitrone cyclooctyne reactions: $[3+2]$ cycloadditions.
    \begin{itemize}
        \item Strain-promoted alkyne-nitrone cycloadditions (SPANC).
        \item Uses more reactive 1,3-dipole nitrous in place of azides.
        \item Rate constants up to \SI{60}{\per\molar\per\second} (60-fold faster than SPAAC).
        \item Faster rates means lower reagent concentrations can be used.
        \item Cyclic nitrones are more stable than acyclic counterparts.
    \end{itemize}
    \item Strained-promoted alkene/alkyne-tetrazine reactions (SPATL).
    \begin{itemize}
        \item 1,2,4,5-tetrazines are reacted with electron rich dienophiles (alkenes).
        \item 3,6-diaryl-s-tetrazines were found to be stable in water.
        \item Run in cell media and cell lysate with >80\% yield.
        \item Successfully used to label proteins in vitro and in cells.
    \end{itemize}
    \item Applications of bioorthogonal chemistry.
    \item Introducing ketones with biotin ligase.
    \begin{itemize}
        \item ...
    \end{itemize}
    \item Fluorescent labeling of proteins on ketones.
    \begin{itemize}
        \item Cell surface proteins can be tagged with the AP at their N- and C-termini.
        \item Then attach a ketone moiety, then attach a fluorescent hydrazide.
        \item Labeling is efficient and specific when performed on purified proteins, total cell lysates, and intact cells.
    \end{itemize}
    \item Trafficking of labeled EGFR.
    \begin{itemize}
        \item ...
    \end{itemize}
    \item Zebrafish embryogenesis.
    \begin{itemize}
        \item Answers questions like, "how can you tell the relative levels and types of cell surface glycosylation during the course of embryogenesis and development?"
    \end{itemize}
    \item Using alkyne bioorthogonality to study glycosylation.
    \begin{itemize}
        \item Feed cells \ce{N3}-sugar and see where reactions take place in real time.
        \item Spatio-temporal analysis of glycosylation during embryogenesis and development.
        \item Different fluorophore reagents and different bioorthogonal chemistries let you see the full picture.
    \end{itemize}
\end{itemize}



\section{Final Review Sheet}
\begin{itemize}
    \item \marginnote{2/2/24:}Watson-Franklin-Crick vs. Wobble interactions.
    \begin{itemize}
        \item The reason some codons are the same is in case normal Watson-Franklin-Crick interactions get supplanted by Wobble interactions, we don't want the amino acid paired to change.
    \end{itemize}
    \item At physiological pH, only phosphates are charged.
    \item We can get tautomerization among nucleoside bases; think keto-enol.
    \item Puckomers: Ribose conformers.
    \item Why phosphate is cool: Multivalent, cannot cross biological barriers, kinetically stable to hydrolysis, thermodynamically unstable, kinetically unstable with catalyst.
    \item A-, B-, and Z-DNA.
    \item G-quadruplexes.
    \item Tagging proteins with GFP, and RNA with the spinach aptamer.
    \item DNA binding proteins.
    \begin{itemize}
        \item Interact w/ DNA major + minor groves.
        \item Leucuine zipper and zinc finger; precursors to CRISPR.
        \item Intercalators: Like the toxic molecule ethidium bromide.
        \begin{itemize}
            \item Design safer ones by making bigger molecules; ones that still intercalate DNA but don't penetrate the skin as easily.
        \end{itemize}
    \end{itemize}
    \item DNA replicase.
    \begin{itemize}
        \item One \ce{Mg^2+} stabilizes the dNTP's two extra phosphate groups; the other stabilizes the acyl substitution intermediate.
        \item Repairs the strand by sliding back to check previous bases before moving on.
    \end{itemize}
    \item The causes of mutations.
    \begin{itemize}
        \item Natural mismatching and tautomerization.
        \item Deamination of exocyclic amines.
        \item Depurination and cleavage.
    \end{itemize}
    \item Four strategies of DNA repair.
    \begin{itemize}
        \item Direct reversal/repair: Enzymes catalyze the reverse reaction of whatever nefarious transformation (e.g., deamination) took place.
        \item Base excision repair: Take out a base, put in a new one.
        \item Nucleotide excision repair: Take out a string of nucleotides, synthesize a new one.
        \item Mismatch repair: Take out a mismatch on the \emph{unmethylated} strand, resynthesize.
    \end{itemize}
    \item Bioorthogonal chemistry and the click reaction (azide plus alkyne).
    \item Promoter regions (for RNA synthesis).
    \item Nucleic acids may have catalyzed reactions in early forms of life.
    \item \emph{Tetrahymena} Catalytic RNA.
    \item Mechanism of the ribosome: A2486 catalyzes protein formation via proton transfer.
    \item Bisulfite chemistry to detect methylated cystine.
    \begin{itemize}
        \item Bisulfite plus heat converts cystine to uracil, so sequence once on its own and once after bisulfite chemistry and look for which base pairs don't change.
    \end{itemize}
    \item Almost all amino acids are in their chiral L-form.
    \begin{itemize}
        \item We can build D-proteins, though.
        \item These cannot be degraded by natural protease and hence are much more stable.
    \end{itemize}
    \item Native chemical ligation: Connecting two peptides with an amide bond.
    \item Know a bit about how to draw nucleosides, and the amino acids.
    \begin{itemize}
        \item Achiral.
        \begin{itemize}
            \item Glycine: Flexible. GGS linkers are nice.
        \end{itemize}
        \item Hydrophobic.
        \begin{itemize}
            \item Ala/A: Simple; good to mutate things to to determine importance. Often works as an inert filler.
            \item V,L,I: Use these for bulk, e.g., to decrease the size of active sites and make pockets smaller.
            \item M: Good start codon. Part of the cofactor SAM, a methyl donor.
            \item Proline: Inflexible.
            \item Phe/F, Tyr/Y, Trp/W aren't too interesting.
        \end{itemize}
        \item Charged.
        \begin{itemize}
            \item Acids deprotonated and bases protonated at physiological pH.
            \item Asp/D, Glu/E, Lys/K, and Arg/R aren't too interesting.
            \item His/H is a great base/acid for a proton shuffle.
        \end{itemize}
        \item Polar.
        \begin{itemize}
            \item Ser/S can be phosphorylated (just like Y). Often a nucleophile in protein active sites.
            \item Thr/T, unimportant.
            \item Cys/C forms disulfide bridges, unlike M.
            \item Asn/N is often a metal coordinate (think \ce{Mg^2+} ions in DNA polymerase!).
            \item Gln/Q is similar to N.
        \end{itemize}
    \end{itemize}
    \item Protein structure in 4 levels: Primary, secondary, tertiary, and quaternary.
    \item Serine protease.
    \begin{itemize}
        \item Asp/D, His/H, and Ser/S work together to cleave a protein.
    \end{itemize}
    \item Structural biology stuff.
    \begin{itemize}
        \item Lots of XRD.
        \item LCLS.
        \item NMR (for in-situ; higher dimensional).
        \item CryoEM (for big stuff).
        \item AlphaFold 2.
    \end{itemize}
    \item PCR and thermal cycling with primer.
    \item DNA sequencing.
    \begin{itemize}
        \item Maxam-Gilbert.
        \begin{itemize}
            \item This is the one with variable cleavage reagents and separation in a gel.
        \end{itemize}
        \item Sanger.
        \begin{itemize}
            \item Spike a bit of ddNTP into cloning bath to stop cloning at a certain point.
            \item Either do sequential (ddATP, ddGTP, \dots) or parallel (ddNTPs plus fluorophore).
        \end{itemize}
        \item Pyro (454).
        \begin{itemize}
            \item Create a bunch of copies, immobilized via biotin and streptavidin beads.
            \item Add a specific type of dNTP to be incorporated, releasing a PPi, which gets converted to ATP and a flash of light that can be detected by two consecutive enzymes.
            \item Get rid of excess dNTP and repeat.
            \item You can do this in a well plate to run many tests in parallel.
            \item Important for neanderthal genome.
        \end{itemize}
        \item Illumina.
        \begin{itemize}
            \item Currently the most important.
            \item Bridging boi.
            \item Read each strand individually by flashes of light, though there will be multiple copies in each cluster, so multiple photons will be released.
            \item We then sort through the data to align all pairs.
        \end{itemize}
        \item SMRT.
        \begin{itemize}
            \item Fix DNA polymerase in a zero-mode waveguide (tiny pore), run a sequence through it recording flashes as each dNTP + fluorophore is added.
            \item DNA polymerase is slowed down to a recordable speed by attaching a protecting group to each dNTP's $3'$ site.
        \end{itemize}
        \item Nanopore.
        \begin{itemize}
            \item Still in development.
            \item Electrical rather than chemical.
            \item A single strand passes through a pore; each dNTP blocks ion flow through the pore to a different, unique, detectable extent.
            \item Allows for direct sequencing of methylated bases, too!
        \end{itemize}
    \end{itemize}
    \item The fluid mosaic model of the plasma membrane is \emph{very wrong}.
    \begin{itemize}
        \item 500-2000 different kinds of lipid molecules.
        \begin{itemize}
            \item Different alkyl chain lengths and degrees of unsaturation.
            \item Different head groups, too.
        \end{itemize}
        \item 17-23\% cholesterol.
        \item Asymmetric; different things face in vs. out.
        \item There are many different ways a protein can stick into the plasma membrane.
        \begin{itemize}
            \item Single-pass (via an amphipathic helix), multi-pass, lipid anchoring, GPI anchoring, $\beta$-barrel proteins.
            \item Predict transmembrane domains with the hydropathy index.
        \end{itemize}
        \item Know when a cell has died via annexin staining; PS phospholipids are constantly flipped inward while the cell is alive, but when it dies, it can no longer do this, signalling to immune cells that it is dead.
        \item Extracellular proteins can bend the plasma membrane.
    \end{itemize}
    \item There are various kinds of transporters and transportation.
    \item Chaperones bring misfolded proteins to the right place and allow them to fold correctly.
    \item Remember topologically equivalent compartments.
    \item Isolating organelles by progressively fast centrifugation.
    \item N-terminus localization sequence for proteins.
    \item Nuclear pores.
    \begin{itemize}
        \item Membrane ring proteins to bend nuclear membrane around pores.
        \item Cytosolic fibrils and nuclear basket.
        \item Nuclear import adapter proteins.
        \item Ran-GTP/Ran-GAP in and Ran-GDP/Ran-GEF out.
    \end{itemize}
    \item Mitochondrial transport.
    \begin{itemize}
        \item TOM: Outer to interluminal.
        \item TIM: Interluminal to inner.
        \begin{itemize}
            \item Locks together with TOM to do outer to inner directly.
            \item Translocation sequences get cleaved once inside inner matrix.
            \item Inner membrane protein? If TIM encounters a stop transfer sequence.
            \begin{itemize}
                \item Multipass? Stop transfer sequence in the middle.
            \end{itemize}
        \end{itemize}
        \item SAM: Insertion into outer membrane.
        \begin{itemize}
            \item Takes proteins brought to interluminal space by TOM.
        \end{itemize}
        \item OXA: Interluminal to inner.
        \begin{itemize}
            \item Can do in inner membrane if it encounters a stop transfer sequence in a protein in the matrix.
        \end{itemize}
        \item To get into the interluminal space, either insert into inner membrane and cleave w/ signal peptidase \emph{or} get in through TOM and fix there with Mia40 (forming disulfide bridges between different proteins that prevent it moving further into the matrix).
    \end{itemize}
    \item Getting proteins into ER lumen.
    \begin{itemize}
        \item SRP grabs ribosome with growing peptide with signal sequence, brings it to ER membrane receptor, portein grows into ER lumen and is left there.
    \end{itemize}
    \item Getting proteins into the ER membrane.
    \begin{itemize}
        \item Snare proteins grab C-terminus localization sequences. Get1, Get2, Get3 over to ER membrane and insertion.
    \end{itemize}
    \item GPI anchors.
    \begin{itemize}
        \item Lipids to phosphate to many sugars to protein.
    \end{itemize}
    \item Golgi.
    \begin{itemize}
        \item Stacks are cisterni.
        \item Golgi cisternae kind of grow up. Either the \textbf{maturation model} or the \textbf{vesicle transport model}.
    \end{itemize}
    \item RNA interference.
    \begin{itemize}
        \item Degradation of mRNA triggered by homologous dsRNA.
        \item Something with double- vs. single-stranded RNA.
    \end{itemize}
    \item Unnnatural amino acid incorporation.
    \begin{itemize}
        \item CRISPR-Cas9, guide RNA, selective cleaving and then cellular repair integrates what you want.
        \item Unnatural amino acid incorporation via bioorthogonal tRNA to allow you to control protein structure.
    \end{itemize}
\end{itemize}




\end{document}